% Generated by Sphinx.
\def\sphinxdocclass{report}
\documentclass[letterpaper,10pt]{sphinxmanual}
\usepackage[utf8]{inputenc}
\DeclareUnicodeCharacter{00A0}{\nobreakspace}
\usepackage{cmap}
\usepackage[T1]{fontenc}
\usepackage{babel}
\usepackage{times}
\usepackage[Sonny]{fncychap}
\usepackage{longtable}
\usepackage{sphinx}
\usepackage{multirow}
\usepackage{eqparbox}


\addto\captions{\renewcommand{\figurename}{Fig. }}
\addto\captions{\renewcommand{\tablename}{Table }}
\SetupFloatingEnvironment{literal-block}{name=Listing }



\title{bambus3 Documentation}
\date{March 06, 2017}
\release{1.0}
\author{Jay Ghurye}
\newcommand{\sphinxlogo}{}
\renewcommand{\releasename}{Release}
\setcounter{tocdepth}{1}
\makeindex

\makeatletter
\def\PYG@reset{\let\PYG@it=\relax \let\PYG@bf=\relax%
    \let\PYG@ul=\relax \let\PYG@tc=\relax%
    \let\PYG@bc=\relax \let\PYG@ff=\relax}
\def\PYG@tok#1{\csname PYG@tok@#1\endcsname}
\def\PYG@toks#1+{\ifx\relax#1\empty\else%
    \PYG@tok{#1}\expandafter\PYG@toks\fi}
\def\PYG@do#1{\PYG@bc{\PYG@tc{\PYG@ul{%
    \PYG@it{\PYG@bf{\PYG@ff{#1}}}}}}}
\def\PYG#1#2{\PYG@reset\PYG@toks#1+\relax+\PYG@do{#2}}

\expandafter\def\csname PYG@tok@gd\endcsname{\def\PYG@tc##1{\textcolor[rgb]{0.63,0.00,0.00}{##1}}}
\expandafter\def\csname PYG@tok@gu\endcsname{\let\PYG@bf=\textbf\def\PYG@tc##1{\textcolor[rgb]{0.50,0.00,0.50}{##1}}}
\expandafter\def\csname PYG@tok@gt\endcsname{\def\PYG@tc##1{\textcolor[rgb]{0.00,0.27,0.87}{##1}}}
\expandafter\def\csname PYG@tok@gs\endcsname{\let\PYG@bf=\textbf}
\expandafter\def\csname PYG@tok@gr\endcsname{\def\PYG@tc##1{\textcolor[rgb]{1.00,0.00,0.00}{##1}}}
\expandafter\def\csname PYG@tok@cm\endcsname{\let\PYG@it=\textit\def\PYG@tc##1{\textcolor[rgb]{0.25,0.50,0.56}{##1}}}
\expandafter\def\csname PYG@tok@vg\endcsname{\def\PYG@tc##1{\textcolor[rgb]{0.73,0.38,0.84}{##1}}}
\expandafter\def\csname PYG@tok@vi\endcsname{\def\PYG@tc##1{\textcolor[rgb]{0.73,0.38,0.84}{##1}}}
\expandafter\def\csname PYG@tok@mh\endcsname{\def\PYG@tc##1{\textcolor[rgb]{0.13,0.50,0.31}{##1}}}
\expandafter\def\csname PYG@tok@cs\endcsname{\def\PYG@tc##1{\textcolor[rgb]{0.25,0.50,0.56}{##1}}\def\PYG@bc##1{\setlength{\fboxsep}{0pt}\colorbox[rgb]{1.00,0.94,0.94}{\strut ##1}}}
\expandafter\def\csname PYG@tok@ge\endcsname{\let\PYG@it=\textit}
\expandafter\def\csname PYG@tok@vc\endcsname{\def\PYG@tc##1{\textcolor[rgb]{0.73,0.38,0.84}{##1}}}
\expandafter\def\csname PYG@tok@il\endcsname{\def\PYG@tc##1{\textcolor[rgb]{0.13,0.50,0.31}{##1}}}
\expandafter\def\csname PYG@tok@go\endcsname{\def\PYG@tc##1{\textcolor[rgb]{0.20,0.20,0.20}{##1}}}
\expandafter\def\csname PYG@tok@cp\endcsname{\def\PYG@tc##1{\textcolor[rgb]{0.00,0.44,0.13}{##1}}}
\expandafter\def\csname PYG@tok@gi\endcsname{\def\PYG@tc##1{\textcolor[rgb]{0.00,0.63,0.00}{##1}}}
\expandafter\def\csname PYG@tok@gh\endcsname{\let\PYG@bf=\textbf\def\PYG@tc##1{\textcolor[rgb]{0.00,0.00,0.50}{##1}}}
\expandafter\def\csname PYG@tok@ni\endcsname{\let\PYG@bf=\textbf\def\PYG@tc##1{\textcolor[rgb]{0.84,0.33,0.22}{##1}}}
\expandafter\def\csname PYG@tok@nl\endcsname{\let\PYG@bf=\textbf\def\PYG@tc##1{\textcolor[rgb]{0.00,0.13,0.44}{##1}}}
\expandafter\def\csname PYG@tok@nn\endcsname{\let\PYG@bf=\textbf\def\PYG@tc##1{\textcolor[rgb]{0.05,0.52,0.71}{##1}}}
\expandafter\def\csname PYG@tok@no\endcsname{\def\PYG@tc##1{\textcolor[rgb]{0.38,0.68,0.84}{##1}}}
\expandafter\def\csname PYG@tok@na\endcsname{\def\PYG@tc##1{\textcolor[rgb]{0.25,0.44,0.63}{##1}}}
\expandafter\def\csname PYG@tok@nb\endcsname{\def\PYG@tc##1{\textcolor[rgb]{0.00,0.44,0.13}{##1}}}
\expandafter\def\csname PYG@tok@nc\endcsname{\let\PYG@bf=\textbf\def\PYG@tc##1{\textcolor[rgb]{0.05,0.52,0.71}{##1}}}
\expandafter\def\csname PYG@tok@nd\endcsname{\let\PYG@bf=\textbf\def\PYG@tc##1{\textcolor[rgb]{0.33,0.33,0.33}{##1}}}
\expandafter\def\csname PYG@tok@ne\endcsname{\def\PYG@tc##1{\textcolor[rgb]{0.00,0.44,0.13}{##1}}}
\expandafter\def\csname PYG@tok@nf\endcsname{\def\PYG@tc##1{\textcolor[rgb]{0.02,0.16,0.49}{##1}}}
\expandafter\def\csname PYG@tok@si\endcsname{\let\PYG@it=\textit\def\PYG@tc##1{\textcolor[rgb]{0.44,0.63,0.82}{##1}}}
\expandafter\def\csname PYG@tok@s2\endcsname{\def\PYG@tc##1{\textcolor[rgb]{0.25,0.44,0.63}{##1}}}
\expandafter\def\csname PYG@tok@nt\endcsname{\let\PYG@bf=\textbf\def\PYG@tc##1{\textcolor[rgb]{0.02,0.16,0.45}{##1}}}
\expandafter\def\csname PYG@tok@nv\endcsname{\def\PYG@tc##1{\textcolor[rgb]{0.73,0.38,0.84}{##1}}}
\expandafter\def\csname PYG@tok@s1\endcsname{\def\PYG@tc##1{\textcolor[rgb]{0.25,0.44,0.63}{##1}}}
\expandafter\def\csname PYG@tok@ch\endcsname{\let\PYG@it=\textit\def\PYG@tc##1{\textcolor[rgb]{0.25,0.50,0.56}{##1}}}
\expandafter\def\csname PYG@tok@m\endcsname{\def\PYG@tc##1{\textcolor[rgb]{0.13,0.50,0.31}{##1}}}
\expandafter\def\csname PYG@tok@gp\endcsname{\let\PYG@bf=\textbf\def\PYG@tc##1{\textcolor[rgb]{0.78,0.36,0.04}{##1}}}
\expandafter\def\csname PYG@tok@sh\endcsname{\def\PYG@tc##1{\textcolor[rgb]{0.25,0.44,0.63}{##1}}}
\expandafter\def\csname PYG@tok@ow\endcsname{\let\PYG@bf=\textbf\def\PYG@tc##1{\textcolor[rgb]{0.00,0.44,0.13}{##1}}}
\expandafter\def\csname PYG@tok@sx\endcsname{\def\PYG@tc##1{\textcolor[rgb]{0.78,0.36,0.04}{##1}}}
\expandafter\def\csname PYG@tok@bp\endcsname{\def\PYG@tc##1{\textcolor[rgb]{0.00,0.44,0.13}{##1}}}
\expandafter\def\csname PYG@tok@c1\endcsname{\let\PYG@it=\textit\def\PYG@tc##1{\textcolor[rgb]{0.25,0.50,0.56}{##1}}}
\expandafter\def\csname PYG@tok@o\endcsname{\def\PYG@tc##1{\textcolor[rgb]{0.40,0.40,0.40}{##1}}}
\expandafter\def\csname PYG@tok@kc\endcsname{\let\PYG@bf=\textbf\def\PYG@tc##1{\textcolor[rgb]{0.00,0.44,0.13}{##1}}}
\expandafter\def\csname PYG@tok@c\endcsname{\let\PYG@it=\textit\def\PYG@tc##1{\textcolor[rgb]{0.25,0.50,0.56}{##1}}}
\expandafter\def\csname PYG@tok@mf\endcsname{\def\PYG@tc##1{\textcolor[rgb]{0.13,0.50,0.31}{##1}}}
\expandafter\def\csname PYG@tok@err\endcsname{\def\PYG@bc##1{\setlength{\fboxsep}{0pt}\fcolorbox[rgb]{1.00,0.00,0.00}{1,1,1}{\strut ##1}}}
\expandafter\def\csname PYG@tok@mb\endcsname{\def\PYG@tc##1{\textcolor[rgb]{0.13,0.50,0.31}{##1}}}
\expandafter\def\csname PYG@tok@ss\endcsname{\def\PYG@tc##1{\textcolor[rgb]{0.32,0.47,0.09}{##1}}}
\expandafter\def\csname PYG@tok@sr\endcsname{\def\PYG@tc##1{\textcolor[rgb]{0.14,0.33,0.53}{##1}}}
\expandafter\def\csname PYG@tok@mo\endcsname{\def\PYG@tc##1{\textcolor[rgb]{0.13,0.50,0.31}{##1}}}
\expandafter\def\csname PYG@tok@kd\endcsname{\let\PYG@bf=\textbf\def\PYG@tc##1{\textcolor[rgb]{0.00,0.44,0.13}{##1}}}
\expandafter\def\csname PYG@tok@mi\endcsname{\def\PYG@tc##1{\textcolor[rgb]{0.13,0.50,0.31}{##1}}}
\expandafter\def\csname PYG@tok@kn\endcsname{\let\PYG@bf=\textbf\def\PYG@tc##1{\textcolor[rgb]{0.00,0.44,0.13}{##1}}}
\expandafter\def\csname PYG@tok@cpf\endcsname{\let\PYG@it=\textit\def\PYG@tc##1{\textcolor[rgb]{0.25,0.50,0.56}{##1}}}
\expandafter\def\csname PYG@tok@kr\endcsname{\let\PYG@bf=\textbf\def\PYG@tc##1{\textcolor[rgb]{0.00,0.44,0.13}{##1}}}
\expandafter\def\csname PYG@tok@s\endcsname{\def\PYG@tc##1{\textcolor[rgb]{0.25,0.44,0.63}{##1}}}
\expandafter\def\csname PYG@tok@kp\endcsname{\def\PYG@tc##1{\textcolor[rgb]{0.00,0.44,0.13}{##1}}}
\expandafter\def\csname PYG@tok@w\endcsname{\def\PYG@tc##1{\textcolor[rgb]{0.73,0.73,0.73}{##1}}}
\expandafter\def\csname PYG@tok@kt\endcsname{\def\PYG@tc##1{\textcolor[rgb]{0.56,0.13,0.00}{##1}}}
\expandafter\def\csname PYG@tok@sc\endcsname{\def\PYG@tc##1{\textcolor[rgb]{0.25,0.44,0.63}{##1}}}
\expandafter\def\csname PYG@tok@sb\endcsname{\def\PYG@tc##1{\textcolor[rgb]{0.25,0.44,0.63}{##1}}}
\expandafter\def\csname PYG@tok@k\endcsname{\let\PYG@bf=\textbf\def\PYG@tc##1{\textcolor[rgb]{0.00,0.44,0.13}{##1}}}
\expandafter\def\csname PYG@tok@se\endcsname{\let\PYG@bf=\textbf\def\PYG@tc##1{\textcolor[rgb]{0.25,0.44,0.63}{##1}}}
\expandafter\def\csname PYG@tok@sd\endcsname{\let\PYG@it=\textit\def\PYG@tc##1{\textcolor[rgb]{0.25,0.44,0.63}{##1}}}

\def\PYGZbs{\char`\\}
\def\PYGZus{\char`\_}
\def\PYGZob{\char`\{}
\def\PYGZcb{\char`\}}
\def\PYGZca{\char`\^}
\def\PYGZam{\char`\&}
\def\PYGZlt{\char`\<}
\def\PYGZgt{\char`\>}
\def\PYGZsh{\char`\#}
\def\PYGZpc{\char`\%}
\def\PYGZdl{\char`\$}
\def\PYGZhy{\char`\-}
\def\PYGZsq{\char`\'}
\def\PYGZdq{\char`\"}
\def\PYGZti{\char`\~}
% for compatibility with earlier versions
\def\PYGZat{@}
\def\PYGZlb{[}
\def\PYGZrb{]}
\makeatother

\renewcommand\PYGZsq{\textquotesingle}

\begin{document}

\maketitle
\tableofcontents
\phantomsection\label{index::doc}


\begin{DUlineblock}{0em}
\item[] 
\end{DUlineblock}

Metagenome scaffolding using mate pair data

\begin{DUlineblock}{0em}
\item[] 
\end{DUlineblock}


\chapter{Documentation}
\label{index:welcome-to-bambus3-s-documentation}\label{index:documentation}

\section{Bambus3 Pipeline}
\label{tutorial:bambus3-pipeline}\label{tutorial::doc}
Bambus3 is a scaffolding program designed specifically to address the issues concerning the metagenomic data.
The scaffolding algorithm consists of several steps. Bambus3 requires contig assembly fasta file, mapping of reads to contigs in BAM file and output directory as an input. It first analyzes the read to contig mapping and estimates the library insert size. Using these mappings, it generates a scaffold graph where nodes are contigs and edges are mate pairs mapped to two different contigs. This is done in \code{libcorrect} program. Multiple links between two contigs are collapsed into a single link using \code{bundler} program. After this, repeats in the graph are removed. High betweenness centrality nodes are identified using \code{centrality} program and removed from the graph using \code{repeat\_filter.py} program. Contigs remaining after filtering repetitive contigs are assigned a unique orientation using \code{orientcontigs} program. The bubbles caused due to polymorphic regions in the metagenome are identified using \code{spqr} program. Once these bubbles are identified, they are collapsed and contig layout is performed in \code{layout.py} program.


\section{Preparing the data}
\label{tutorial:preparing-the-data}
In order to run Bambus3, you will need two files. First one is the contig assembly of the raw reads using your favorite metagenome assembly program.
The second one is the alignment of raw paired end reads to assembled contigs using your favorite alignment program. We strongly recommend to map paired end reads as single end reads since aligner enforces the reads to map in the distances given by the library size. If you are using Bowtie2 for read mapping, we recommend aligning reads this way:

\begin{Verbatim}[commandchars=\\\{\}]
bowtie2 \PYGZhy{}x idx \PYGZhy{}U forward\PYGZus{}reads.fq \textbar{} samtools view \PYGZhy{}bS \PYGZgt{} alignment\PYGZus{}1.bam
bowtie2 \PYGZhy{}x idx \PYGZhy{}U reverse\PYGZus{}reads.fq \textbar{} samtools view \PYGZhy{}bS \PYGZgt{} alignment\PYGZus{}2.bam
samtools alignment\PYGZus{}total.bam alignment\PYGZus{}1.bam alignment\PYGZus{}2.bam
samtools sort \PYGZhy{}n alignment\PYGZus{}total.bam \PYGZhy{}o alignment\PYGZus{}total\PYGZus{}sorted.bam
\end{Verbatim}

We need this file sorted by the read name and not the coordinates. You will also need to provide the output directory where you want your scaffolds to be written.


\section{Running Bambus3}
\label{tutorial:running-bambus3}
Before you run Bambus3, please make sure you have following dependencies: \href{http://bowtie-bio.sourceforge.net/manual.shtml}{samtools}, \href{http://bedtools.readthedocs.io/en/latest/}{bedtools} and \href{https://networkx.github.io/}{NetworkX}. After you have these dependencies, running Bambus3 is pretty straightforward. You would need to execute \code{run.py} file.

\begin{Verbatim}[commandchars=\\\{\}]
python run.py \PYGZhy{}h
usage: run.py [\PYGZhy{}h] \PYGZhy{}a ASSEMBLY \PYGZhy{}m MAPPING \PYGZhy{}d DIR [\PYGZhy{}f FORCE] [\PYGZhy{}r REPEATS]
              [\PYGZhy{}k KEEP] [\PYGZhy{}l LENGTH] [\PYGZhy{}b BSIZE]

Bambus3: A scaffolding tool for metagenomic assemblies

optional arguments:
  \PYGZhy{}h, \PYGZhy{}\PYGZhy{}help            show this help message and exit
  \PYGZhy{}a ASSEMBLY, \PYGZhy{}\PYGZhy{}assembly ASSEMBLY
                        assembled contigs
  \PYGZhy{}m MAPPING, \PYGZhy{}\PYGZhy{}mapping MAPPING
                        mapping of read to contigs in bam format
  \PYGZhy{}d DIR, \PYGZhy{}\PYGZhy{}dir DIR     output directory for results
  \PYGZhy{}f FORCE, \PYGZhy{}\PYGZhy{}force FORCE
                        force re\PYGZhy{}run of pipeline, will remove any existing
                        output
  \PYGZhy{}r REPEATS, \PYGZhy{}\PYGZhy{}repeats REPEATS
                        To turn repeat detection on
  \PYGZhy{}k KEEP, \PYGZhy{}\PYGZhy{}keep KEEP  Set this to kepp temporary files in output directory
  \PYGZhy{}l LENGTH, \PYGZhy{}\PYGZhy{}length LENGTH
                        Minimum length of contigs to consider for scaffolding
  \PYGZhy{}b BSIZE, \PYGZhy{}\PYGZhy{}bsize BSIZE
                        Minium mate pair support between contigs to consider
                        for scaffolding
\end{Verbatim}

The mandatory inputs for this are the FASTA file for contig assembly, the alignment file and output directory. There are other options as well.
\code{-r} option lets you toggle the repeat detection process in the scaffolding pipeline. By default it is turned off, but you can provide \code{-r true} parameter to include repeat detection. :code: \emph{-l} option lets you set the minimum length of the contigs to consider for scaffolding. By default this is set to :code: 500. \code{-b} option lets you set the minimum mate pair support between two contigs to consider for scaffolding. By default this is set to \code{3}. Each step in the pipeline generates temporary files as intermediate outputs. If you want to retain these files, you need to set \code{-k} option to \code{true}. By default, these files will be removed.


\section{Running Bambus3}
\label{tutorial:id1}


\renewcommand{\indexname}{Index}
\printindex
\end{document}
